\section{W2: Processes, threads}
\textbf{Why processes?} Simultaneous execution of multiple programs on single CPU.\\
\textbf{Why threads?} Simultaneous execution of multiple parts of a program on single CPU. Threads share global variables and heap memory.\\

\subsection{Interprocess Communication}
\textit{Process/Thread Synchronization} Race conditions occur when multiple processes/threads access shared data and try to change it at the same time.\\
\textbf{Critical region:} code segment where shared data is accessed.\\
\textbf{Mutual exclusion:} only one process/thread can be in critical region at a time.\\
\begin{enumerate}
    \item No two processes/threads may be simultaneously inside their critical regions.
    \item No assumptions may be made about speeds or the number of CPUs.
    \item No process/thread running outside its critical region may block other processes/threads.
    \item No process/thread should have to wait forever to enter its critical region.
\end{enumerate}

\subsubsection{Possible approaches to mutual exclusion}
\textbf{Busy waiting:} process/thread stays in loop in kernel mode until it can enter critical region. Pros: simple, no context switch. Cons: wastes CPU time, starvation (Priority Inversion Problem).\\
\begin{enumerate}
    \item \textbf{Lock variables:} each process/thread has a lock variable. Process/thread can enter critical region only if its lock variable is 0.\\
    \item \textbf{Strict alternation:} processes/threads take turns entering critical region.\\
    \item \textbf{Peterson's solution:} two processes/threads, two lock variables.\\
    \item \textbf{Test-and-set (TSL):} atomic instruction that sets lock variable to 1 and returns old value.\\
\end{enumerate}

\subsubsection{Deadlocks}
Four conditions for deadlock:
\begin{enumerate}
    \item \textbf{Mutual exclusion:} only one process/thread can use a resource at a time.
    \item \textbf{Hold and wait:} processes/threads hold resources while waiting for others.
    \item \textbf{No preemption:} resources can be released only voluntarily by process/thread holding it.
    \item \textbf{Circular wait:} two or more processes/threads form a circular chain where each process/thread waits for a resource held by next process/thread in chain.
\end{enumerate}