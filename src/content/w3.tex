\section{W3: Scheduling, Memory Management}
\textbf{Preemption:} Preemptive scheduling is when the scheduler decides to switch to another process/thread. Clock interrupts used upon a quantum. Non-preemptive scheduling is when the process/thread itself decides to give up the CPU or is blocked.\\
\textbf{Context switch:} Switching from one process/thread to another is expensive.\\

\subsection{Non-preemptive scheduling}
\textbf{First-come, first-served (FCFS):} Simple, but long average waiting time.\\
\textbf{Shortest job first (SJF):} Shortest average waiting time, but long waiting time for long processes/threads. Average turnaround time is minimized.\\

\subsection{Preemptive scheduling}
\textbf{Round-robin (RR):} Each process/thread gets a quantum.\\
\textbf{Shortest remaining time first (SRTF):} Shortest average waiting time, but long waiting time for long processes/threads. Average turnaround time is minimized.\\
\textbf{Priority scheduling:} Each process/thread has a priority. May lead to starvation.\\

\subsection{Memory management}
\textbf{Why Memory Management?} Support multiprogramming. Secure isolation between processes/threads and OS memory. Enables virtual memory which allows processes/threads to use more memory than is physically available.\\
\textbf{No memory abstraction:} Use swap space on disk. Multiplex physical memory.\\
\textbf{Memory abstraction:} 